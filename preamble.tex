% dmthesis.sty : LaTeX style for my Master's Theses
% created by :
%    Dan Michael Olsen Heggø
%    (danmichaelo@gmail.com)

%\typeout{Hello world}

\KOMAoptions{parskip=half, BCOR=7.5mm, DIV=11, mpinclude=false}  % skip margin notes
% DIV=11 gir oss i utg. pkt. 152.73 mm tekstbredde og 216 mm teksthøyde.
% men vi trekker fra 7.5 mm som forsvinner i innbindingen (BCOR)
% og står igjen med 147.26 mm tekstbredde.

\usepackage[greek,british]{babel}		% Localization
\usepackage[utf8]{inputenc}				% For typing accentuated characters directly

%\usepackage{kpfonts}                   % kp is nice for text, but the math isn't acceptable 
\usepackage{lmodern}                    % Use fully scalable fonts
\usepackage[T1]{fontenc}            	% T1 contains the french guillements.. 
\usepackage{lettrine}

%%%%%%%%%%%%%%%%%%%%%%%%%%%%%%%%%%%
% Page headers and footers:
\usepackage{scrpage2}  % part of koma-script
\ohead{\pagemark}   % outer
\rehead{\headmark}  % right even
\lohead{\headmark}  % left odd
\ofoot{} % clear default
\automark[section]{chapter}  % right, left
%\setheadtopline{2pt}
%\setheadsepline{.4pt}


%%%%%%%%%%%%%%%%%%%%%%%%%%%%%%%%%%%
% MATH:

\usepackage{amsmath,amsthm,amssymb}		% AMS Math support
\usepackage{empheq,mathtools}
\usepackage{siunitx} 					% For consistent units. Includes upright mu.
										% Must be loaded after amssymb

%%%%%%%%%%%%%%%%%%%%%%%%%%%%%%%%%%%
% FONT:

%\usepackage{kmath,kerkis}          	% The order of the packages matters; 
                                    	% kmath changes the default text font
                                    	% Must be loaded after amsmath

%%%%%%%%%%%%%%%%%%%%%%%%%%%%%%%%%%%
% Figures and tables:

\usepackage{color}	% usenames + dvipsnames gives us 66 predefined colors. (not needed anymore?)
\usepackage{graphicx}					% Graphicx is the extended graphics package.
%\graphicspath{{imgs/}{moreimgs/}}
\DeclareGraphicsExtensions{.pdf,.png,.jpg} 	% specifies the behaviour of the system when no 
											% file extension is specified in the argument
\usepackage{subfig} 					% replacement for the older subfigure package

%\usepackage{wrapfig} 					% for \begin{wrapfigure}

% Print figure and table labels in a smaller font, and print `Figure:' and `Table:' in boldface:
\usepackage[margin=10pt,font=small,labelfont=bf,labelsep=colon,format=plain,indention=.5cm]{caption}


\usepackage{booktabs}					% production quality tables
\usepackage{threeparttable}				% for placing footnotes under tables
%\usepackage{multirow} 					% for multi-row table cells

\usepackage{framed} % for drafting only

%%%%%%%%%%%%%%%%%%%%%%%%%%%%%%%%%%%
% CHEMISTRY :

%\usepackage[version=3]{mhchem}			% Latest version: 3.07


%%%%%%%%%%%%%%%%%%%%%%%%%%%%%%%%%%%
% MISC :

\usepackage{fancybox}					% For boxed environments
\usepackage{layout}						% For page layout debug (\layout)
\usepackage{soul}                       % For highlighting (in the draft phase)


% Note that LaTeX can only draw lines with slope = x/y, where x and y have integer values from −6 through 6.
\newcommand{\crossbox}[1]{ %
    \setlength{\unitlength}{#1}
    \begin{picture}(1,1)(0,0)
       \put(0,0){\framebox(1,1){ }}
       \put(0,0){\line(1,1){1}}         % (slope x, slope y){length}
       \put(1,0){\line(-1,1){1}}        % (slope x, slope y){length}
    \end{picture} %
    }

\usepackage{pdfpages}                   % to include title page

\usepackage{listings}


%%%%%%%%%%%%%%%%%%%%%%%%%%%%%%%%%%%
% CUSTOM MACROS:

\renewcommand{\vec}[1]{\boldsymbol{\mathrm{#1}}} 	% Use bold vectors
\newcommand{\nvec}[1]{\mathrm{#1}} 	                % matrices, etc.. 

% Numerical constants such as e (base of natural log), i (imaginary unit) should be set in roman
% TUGboat vol. 18 (1997), no. 1, p. 39 <http://www.tug.org/TUGboat/Articles/tb18-1/tb54becc.pdf>
\newcommand{\e}{\mathrm{e}} 
\newcommand{\im}{\mathrm{i}} 

\newcommand{\spl}[1]{split$\;\langle #1 \rangle$}

%%%%%%%%%%%%%%%%%%%%%%%%%%%%%%%%%%%
% BIBLIOGRAPHY :
\usepackage[safeinputenc,backend=biber,hyperref=true,sorting=none,style=numeric]{biblatex}


%%%%%%%%%%%%%%%%%%%%%%%%%%%%%%%%%%%
% HYPERREF :
\usepackage{hyperref} 	% Make sure it comes last of your loaded packages, 
						% to give it a fighting chance of not being over-written, 
						% since its job is to redefine many LATEX commands.
						% It is preferable to load it after biblatex.
\hypersetup{pdftex,unicode, 
    colorlinks=true,linkcolor=blue,citecolor=blue,urlcolor=blue,    
    pdfdisplaydoctitle=true, 
    bookmarksopen=true,bookmarksopenlevel=2,        
    pdfauthor   = {Dan Michael Olsen Heggø},            
    pdftitle    = {Not Set Yet},                         
    pdfkeywords = {phosphorus, diffusion, silicon, solar cells, master thesis}
}


